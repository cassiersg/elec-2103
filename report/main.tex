\documentclass[english, DIV=13]{scrreprt}

\usepackage[utf8x]{inputenc}
\usepackage[T1]{fontenc}
\usepackage{lmodern}
\usepackage{microtype}
\usepackage{xspace}
\usepackage[binary-units=true]{siunitx}
\usepackage{graphicx}
\usepackage{hyperref}
\usepackage{todonotes}
\usepackage{epstopdf}
\usepackage{array}
\usepackage{multicol}
\usepackage{multirow}
\usepackage{tabularx} % tabular with automatic line-break
\newcolumntype{Y}{>{\centering\arraybackslash}X} % centered column
\usepackage{amsmath}
\usepackage{grffile} % better name handling with graphicx
\usepackage{currfile} % provides relative file inclusion for tikzscale
\usepackage{listings}
\lstset{%
    basicstyle=\scriptsize\ttffamily,
    breaklines=true
}

\usepackage{tikz}
\usepackage{pgfplots}
\usepackage{pgfplots}
\usepackage{tikzscale}
\pgfplotsset{compat=newest}
\usetikzlibrary{plotmarks}
\usepackage{rotating}
\usepackage[absolute,overlay]{textpos}
\usepackage{circuitikz}

% Math symbols
\usepackage{amsmath}
\usepackage{amssymb}
\usepackage{amsthm}
\DeclareMathOperator*{\argmin}{arg\,min}
\DeclareMathOperator*{\argmax}{arg\,max}
\newcommand\norm[1]{\left\lVert#1\right\rVert}

% Sets
\newcommand{\Z}{\mathbb{Z}}
\newcommand{\R}{\mathbb{R}}
\newcommand{\Rn}{\R^n}
\newcommand{\Rnn}{\R^{n \times n}}
\newcommand{\C}{\mathbb{C}}
\newcommand{\K}{\mathbb{K}}
\newcommand{\Kn}{\K^n}
\newcommand{\Knn}{\K^{n \times n}}

% Unit vectors
\usepackage{esint}
\usepackage{esvect}
\newcommand{\kmath}{k}
\newcommand{\xunit}{\hat{\imath}}
\newcommand{\yunit}{\hat{\jmath}}
\newcommand{\zunit}{\hat{\kmath}}
\newcommand{\uunit}{\hat{\umath}}

% rot & div & grad & lap
\DeclareMathOperator{\newdiv}{div}
\newcommand{\divn}[1]{\nabla \cdot #1}
\newcommand{\rotn}[1]{\nabla \times #1}
\newcommand{\grad}[1]{\nabla #1}
\newcommand{\gradn}[1]{\nabla #1}
\newcommand{\lap}[1]{\nabla^2 #1}

% Elec
\newcommand{\B}{\vec B}
\newcommand{\E}{\vec E}
\newcommand{\EMF}{\mathcal{E}}
\newcommand{\perm}{\varepsilon} % permittivity

\newcommand{\bigoh}{\mathcal{O}}
\newcommand\eqdef{\triangleq}

\DeclareMathOperator{\newdiff}{d} % use \dif instead
\newcommand{\dif}{\newdiff\!}
\newcommand{\fpart}[2]{\frac{\partial #1}{\partial #2}}
\newcommand{\ffpart}[2]{\frac{\partial^2 #1}{\partial #2^2}}
\newcommand{\fdpart}[3]{\frac{\partial^2 #1}{\partial #2\partial #3}}
\newcommand{\fdif}[2]{\frac{\dif #1}{\dif #2}}
\newcommand{\ffdif}[2]{\frac{\dif^2 #1}{\dif #2^2}}
\newcommand{\constant}{\ensuremath{\mathrm{cst}}}


\title{LELEC2103}
\subtitle{Project in Electricity 3 : Electronic systems}
\author{Gaëtan Cassiers\and Antoine Paris}
\date{\today}

\begin{document}
\maketitle

\chapter{Introduction}

\chapter{Gameplay and features}

\chapter{Network interface}

\begin{figure}
    \centering
    \includegraphics[width=1.0\textwidth]{img/global_comm_scheme_cropped.pdf}
    \caption{Block diagram of the network communication.}
\end{figure}

\chapter{Global view of the client system}

\begin{figure}
    \centering
    \includegraphics[width=0.6\textwidth]{img/block_global_desktop_cropped.pdf}
    \caption{Block diagram of the client, when running on a PC.}
\end{figure}

\begin{figure}
    \centering
    \includegraphics[width=0.6\textwidth]{img/block_global_device_cropped.pdf}
    \caption{Block diagram of the client, when running the raspberry-pi/DE0-Nano system.}
\end{figure}

\chapter{Rendering subsystem}

\begin{figure}
    \centering
    \includegraphics[width=0.6\textwidth]{img/block_rendering_cropped.pdf}
    \caption{Pipeline for the rendering of the video.}
\end{figure}

\chapter{Compression subsystem}

\section{Goal}

The need for a compression system comes from the following observation:
when transferring 25 frames of 800x480 pixel, with 12 bits per pixel,
the bandwidth needed is \SI{115}{Mb/s}. The memory needed for each frame
is \SI{600}{kB}.

In our system, the SPI bandwidth available to transfer the video is \SI{5}{Mb/s},
and the memory available on the FPGA (onchip memory, SDRAM is used by the
nios/µC0S-II system) is \SI{30}{kB} per frame.

We neeed thus a compression of a factor larger than 20, that is such that
the video can be compressed in real-time. The decompression must be done
pixel per pixel, on-the-fly, since there is no memory available to store the
decompressed images.

\section{Principle}

\begin{figure}
    \centering
    \includegraphics[width=0.8\textwidth]{img/compression_scheme_cropped.pdf}
    \caption{Principle view of the compression scheme.}
    \label{fig:compression}
\end{figure}

The compression system is based on three steps (see fig.~\ref{fig:compression}):
the color quantization, the run-length coding, and the Huffman coding. The first
step is lossy, while the two other steps are lossless.

\paragraph{Color quantization} The precision for each color channel is
reduced to 4 bits. This step is a preprocessing step that makes it possible
to implement and improves the efficiency of the following steps.

\paragraph{Run-length coding} The image, represented as a list of pixels,
is converted to a list of pairs $(\text{color}, \text{number of repetitions})$.
The allowed sizes for the number of repetitions are
$1, 2, 3, ..., 16,$ $32, 64, ..., 4096$. This allows to encode most images with
a number of blocks that is not too far from the optimum, while keeping the set
of number of repetitions small, which is required for the Huffman compression.

\paragraph{Huffman coding} Two Huffman trees are built: one for the pixels colors
and one for the numbers of repetitions. Each element of the pairs generated
at the previous stage is encoded with the corresponding Huffman tree.
All the codewords are then concatenated together to form the compressed image.

\section{Impact of the color quantization}

The color quantization has several goals. For the run-length coding,
the quantization creates chunks of constant color when there is a
continuous shading.\todo{Is it a correct wording ?} For the Huffman
coding, the reduction of the pixels coding from \SI{24}{bits} to
\SI{12}{bits} means the size of the colorspace is reduced from
\SI{16}{M} to \SI{4}{K}. This simplifies a lot the Huffman coding:
\begin{itemize}
    \item it is much easier to estimate the distribution of the colors
        to build the Huffman tree\footnote{
            The color and run-length distribution are estimated from a set of training
        images during at compile time.};
    \item it keeps the codewords length small, which simplifies the implementation:
        the memory bandwidth required is smaller;
    \item it reduces the amount of logic gates required to implement a decoder,
        which makes it possible to fit it on the FPGA;
    \item it also makes the software encoder faster, for the same reasons
        (memory bandwidth, code size)
\end{itemize}

The color quantization has an impact of the quality of the images,
but it is mainly visible when there is an area where there are
slightly different shades of the same color. For these kinds of images,
it may introduce a "staircase" effect. To avoid this artefact to be visible,
the rendering process has been tuned to avoid producing continuously varying colors,
through the choice of a adequated lightning scheme for the 3D scene (the lightning
has only the directionnal component of the phong lightning model).

\chapter{SPI slave subsystem}

\begin{figure}
    \centering
    \includegraphics[width=0.5\textwidth]{img/spi_state_machine_cropped.pdf}
    \caption{State machine for the SPI communication protocol.}
\end{figure}

\chapter{Display management}

\begin{figure}
    \centering
    \includegraphics[width=\textwidth]{img/display_manager_cropped.pdf}
    \caption{Block diagram of the display management and synchronization system.}
\end{figure}

\chapter{Input detection subsystem}

\begin{figure}
    \centering
    \includegraphics[width=0.5\textwidth]{img/tasks_mC_cropped.pdf}
    \caption{Tasks and their relationships on the uC/OS-II operating system.}
\end{figure}

\chapter{Player pictures acquisition}

\begin{figure}
    \centering
    \includegraphics[width=0.5\textwidth]{img/block_pictures_cropped.pdf}
    \caption{Block diagram of the picture acquisition subsystem.}
\end{figure}

\chapter{Conclusion}

\end{document}
